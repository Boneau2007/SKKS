\section{MSC}
\label{sc:MSCB}
\ac{MSC} wurde von derselben Arbeitsgruppe wie \ac{SDL} spezifiziert und wird häufig in Verbindung mit anderen Modellierungssprachen verwendet. Dort wird sie unter zur Anforderungsspezifikation, Testfallspezifikation und Dokumentation verwendet. Durch ihre Flexibilität, kann sie in den unterschiedlichsten Anwendungsdomänen verwendet werden, da sie nicht explizit für eine einzige zugeschnitten wurde. Hauptvorteil von \ac{MSC} ist dessen grafisches Aussehen, welches intuitiv die Beschreibung des Systemverhaltens verständlich macht. Der Hauptaugenmerk liegt dabei bei dem Austausch von Nachrichten über Kommunikationseinheiten wie verteilten Systemen.
\subsection{Eignung}
\label{sc:MSCEignung}
\ac{MSC} ist wie \ac{SDL} eine formale Modellierungssprache und deckt damit das Formale Kriterium der Korrektheit aus Abschnitt \pageref{ssc:Korrektheit} ab. 

\subsection{Fazit}
\label{sc:MSCFazit}


MSC wird in folgenden Feldern verwendet
