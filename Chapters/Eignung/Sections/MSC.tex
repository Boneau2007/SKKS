\section{MSC}
\label{sc:MSCB}
\ac{MSC} wurde von derselben Arbeitsgruppe wie \ac{SDL} spezifiziert und wird häufig in Verbindung mit anderen Modellierungssprachen verwendet. Dort wird sie unter zur Anforderungsspezifikation, Testfallspezifikation und Dokumentation verwendet. Durch ihre Flexibilität, kann sie in den unterschiedlichsten Anwendungsdomänen verwendet werden, da sie nicht explizit für eine einzige zugeschnitten wurde. Hauptvorteil von \ac{MSC} ist dessen grafisches Aussehen, welches intuitiv die Beschreibung des Systemverhaltens verständlich macht. Der Hauptaugenmerk liegt dabei bei dem Austausch von Nachrichten über Kommunikationseinheiten wie verteilten Systemen.
\subsection{Eignung}
\label{sc:MSCEignung}
\ac{MSC} ist wie \ac{SDL} eine formale Modellierungssprache und deckt damit das Formale Kriterium der Korrektheit aus Abschnitt \pageref{ssc:Korrektheit} ab. 


We believe that scenario-based requirements will play an increasingly promi-
nent role in design of software systems that require communication among dis-
tributed agents. Requirements expressed using MSCs (or HMSCs) have a formal
semantics, and hence, can be sub jected to analysis. Since MSCs are used at
a very early stage of design, any errors revealed during their analysis have a
high pay-off. This has already motivated development of algorithms for detect-
ing race conditions and timing conflicts [1], pattern matching [15], and detecting
non-local choice [4], and tools such as uBET [1,11]. In this paper, inspired by
the success of model checking in debugging of high-level hardware and software
designs [5,6,10], we develop a methodology and algorithms for model checking
of scenario-based requirements.
\subsection{Uneignung}
\label{sc:MSCUnEignung}


MSC wird in folgenden Feldern verwendet
