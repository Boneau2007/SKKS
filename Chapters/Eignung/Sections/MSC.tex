\section{MSC}
\label{sc:MSCB}
\ac{MSC} wurde von derselben Arbeitsgruppe wie \ac{SDL} spezifiziert und wird häufig in Verbindung mit anderen Modellierungssprachen verwendet. Dort wird sie unter anderem zur Anforderungsspezifikation, Testfallspezifikation und Dokumentation verwendet. Durch ihre Flexibilität, kann sie in den unterschiedlichsten Anwendungsdomänen verwendet werden, da sie nicht explizit für eine einzige zugeschnitten wurde. Hauptvorteil von \ac{MSC} ist dessen grafisches Aussehen, welches intuitiv die Beschreibung des Systemverhaltens verständlich macht. Der Hauptaugenmerk liegt dabei bei dem Austausch von Nachrichten über Kommunikationseinheiten wie in verteilten Systemen.
\subsection{Eignung/Uneignung von MSC}
\label{sc:MSCEignung}
\ac{MSC} ist wie \ac{SDL} eine formale Modellierungssprache und deckt damit unter anderem das Formale Kriterium der Korrektheit  ab. Da auch hier wie in \ac{SDL} eine eindeutig festgelegte und präzise Syntax und Semantik definiert ist, sollte es keinen Interpretationsfreiraum der Konzepte geben und fehlerhafte Modelle erkannt werden.

Auch unter dem Aspekt der Vollständigkeit ist \ac{MSC} durch ihre Beschreibung in der Spezifikation in ihrem Sprachgebrauch vollständig.

Genauso sind die Konzepte und Konstrukte einheitlich definiert und in der Spezifikation beschrieben. Grundsätzlich kommt Redundanz in \ac{MSC} nicht vor dazu kann jede Instanz nur einmal definiert werden und jede Nachricht in genau einem Input und Output enthalten sein.

MSC definiert dabei Instanztypen als Antwort auf eine objektorientierte Modellierung. Sowie Vererbung von Typen. Dadurch Eignet sich der Einsatz von \ac{MSC} mit anderen objektorientierten Programmiersprachen.

Die Sprache ist extrem einfach für den Anwender und Leser zu erlernen da nur äußerst wenig Konstrukte der Sprache definiert sind. So ist ihre Spezifikation gerade mal um die 135 Seiten groß. Zwar wird auch hier wieder auf kosten der Verständlichkeit Fachvokabular aus dem Telekommunikationsbereich verwendet jedoch ist wegen des geringen Sprachumfangs dies für Fachexperten und fachunkundige vernachlässigbar. Symbole sind einfach beschrieben und werden auch im Alltag verwendet.

Da das \ac{MSC} Nachrichtenabläufe zwischen Objekten modelliert, kann es das Interaktionsverhalten zeitlichen zwischen den Objekten anschaulich wiedergeben. Es obliegt jedoch auch hier wieder dem Betrachter ob das betrachtete Modell wirklich anschaulich ist.Ein Chart in \ac{MSC} ist dabei ähnlich anschaulich wie das Sequenzdiagramm von UML.

Auch ist wieder in diesem Fall eine Einschätzung über die Angemessenheit schwierig zu treffen. in Bereichen wie der Protokollspezifizierung enthält die Sprache angemessene Konstrukte jedoch kann dies bei Anwendungsbereichen in denen keine einfache Kommunikation zwischen Objekten vorherrscht sich schwierig gestalten.

Vollständige Systeme können mit \ac{MSC} alleine nicht erstellt werden, dazu sind andere Modellierungssprachen notwendig, welche \ac{MSC} bei der Umsetzung unterstützen. Die Mächtigkeit ist demnach gemessen nicht so hoch wie bei Anderen  Modellierungssprachen, reicht aber für ihren Anwendungszweck vollkommen aus.
Operationalisierbarkeit ist in \ac{MSC} gegeben, so können zum Beispiel beliebige Charts in Kommunikationsdiagrame transformiert werden. Es ist jedoch nicht möglich den Kontext in ein  Strukturdiagramm zu übertragen, da dem Model dazu die nötigen Informationen wie  Beziehungen fehlen.


\subsection{Fazit}
\label{sc:MSCFazit}
Da \ac{MSC} von derselben Arbeitsgruppe erstellt worden ist wie \ac{SDL}, ist die Philosophie zur Erstellung der Modellierungssprache eine ähnliche. Deswegen decken sich viele formale Kriterien mit \ac{SDL}. Da \ac{MSC} oft in Verbindung mit anderen Modellierungssprachen in der Systemtechnik zum erstellen dieser verwendet wird, kann ihr Anwendungsspeckdrum eben auch all jene umfassen. Dazu gehören Elektrotechnik, Mechanik und viele weiter Anwendungsgebiete in der Industrie.

\begin{table}[ht]
	\begin{tabularx}{\textwidth}{|l|X|X|}
		\hline
		Formale	Kriterien &
		\begin{itemize}
			\item Formal beschrieben
			\item Klare, widerspruchsfreie präzise und übersichtlich definierte Konzepte und Konstrukte
			\item Einheitlich Konsistent und die Sprachbausteine sind mit der Spezifikation konform
			\item Objektorientiert und lässt sich gut mit anderen Sprachen verbinden
		\end{itemize}  & 
		\begin{itemize}
			\item fehlen
		\end{itemize} \\
	\hline
	Anwenderbezogene Kriterien &
	\begin{itemize}
		\item Sehr einfach zu erlernen, da extrem kleiner Sprachumfang
		\item Anschauliche Darstellung des zeitlichen Kommunikationsablaufs zweier oder mehreren Objekten
	\end{itemize}  &  \\
		\hline
	Anwendungsbezogene Kriterien &
		\begin{itemize}
			\item Starker Anwendungsbezug zur Telekommunikationsbrache
			\item Kann leicht abgewandelt in UML als Sequenzdiagramm eingesetzt werden
			\item Bedingte Operationalisierbarkeit in andere Diagrammarten
		\end{itemize}  & 
		\begin{itemize}
			\item Vergleichsweise geringe Mächtigkeit zum zu modellierenden Anwendungszweck
			\item Schwierige Einschätzung ob die Modelle Angemessen im Anwendungsbereich sind
			\item Beschreibt lediglich mögliches Systemverhalten, nicht tatsächliches
			\item Es kann kein Verhalten verboten werden
		\end{itemize} \\
		\hline
	\end{tabularx} 
	\caption{Eignung und Uneignung von SDL}
	\label{tab:EignungSDL}
\end{table} 
