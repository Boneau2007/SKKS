\section{SDL}
\label{sc:SDLB}
\ac{SDl} ist eine alte und häufig verwendete Modellierungssprache zum spezifizieren und beschreiben von verteilten Kommunikationssystemen. 
Es schließt die Lücke zwischen Spezifikation und Implementierung, da sie Modellierung auf einem abstraktem Level ermöglicht und eine detaillierte Beschreibung der Implementation liefert. \ac{SDL} verwendet ein  explizites  Nachrichtenkonzept,  welches  zwar  einen  erhöhten  Aufwand  bei  Modellerstellung und -pflege erfordert, dafür aber fehlerrobuster ist. 


formale Sprachen generell haben: Die Erstellung und Pflege von Modellen ist sehr aufwendig. Als alleiniges Beschreibungsmittel sind sie für Systemmodellierungen im Großen daher nicht geeignet.
\subsection{Eignung}

\subsection{Uneignung}

\subsection{Fazit}

\begin{tabular}{|l|c|c|}
	\hline
	Kriterium & Vorteil & Nachteil  \\
	\hline
	Spalte 1 & Aufwendige aber fehlerrobuste Modellierung von Kommunikation durch Nachrichtenkanäle. & Verwendung einer schwer verständlichen formalen axiomatischen Notation. \\
	\hline
	\hline
\end{tabular}