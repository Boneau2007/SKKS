\chapter{Einleitung}
Es gibt in der Welt der Informatik verschiedene Arten von Sprachen zur Beschreibung von Kommunikationsabläufen. So können zum einen beispielsweise Domänenspezifische Sprachen verwendet werden oder globale Modellierungssprachen. Diese genannten Arten von Sprachen dienen zur Abstraktion und zum besseren Verständnis zwischen Mensch und Computer. Sie sind sogenannte Formale Sprachen, welche künstlich erstellte Sprachen sind und sich für eine präzise Beschreibung von Eigenschaften und Verhalten eignen. Das hier vorgestellte Projekt soll ausgwählte Formale Sprachen in hinsicht auf ihre Funktionsweise und  Kommunikationsabläufe kritisch beobachten, beschreiben und vergleichen. Am Ende soll diese Ausarbeitung eine Aussage über die Eignung von Einsatzzwecken der Sprachen bei Kommunikationsabläufen treffen.

\section{Problemstellung}
Da formale Sprachen künstliche Sprachen sind und einen Kontext über eine formale mathematische Syntax beschrieben werden, werden diese bevorzugt zur Abstraktion von Sachverhalten wie der von Kommunikationsabläufen in Systemen verwendet. Nun können unterschiedliche Sprachen einen Sachverhalt unterschiedlich gut oder schlecht wiedergeben. Wobei gut oder schlecht, abhängig des Einsatzortes ist und welche formalen Kriterien eine wiedergabe erleichtert oder erschwert.

\section{Zielsetzung}
In dieser Projektarbeit gilt es den Einsatz unterschiedlicher Sprachen im Einsatz eines Kommunikationssystems zu beschreiben und zu vergleichen. Der Vergleich ist mit eigens gewählten Bewertungskriterien vorzunehmen und über den Einsatzort der Sprachen in einem Kommunikationssystem zu beschreiben. Dazu sollen die Informationen über die ausgewählten Modellierungssprachen aus den Dokumenten der Spezifikationen \acs{SDL} Z.100 [\ref{SDL_Z.100}], \acs{MSC} Z.120 [\ref{MSC_Z.120}] und OMG \acs{UML} [\ref{UML_2.5.1}] entnommen werden. Die Kriterien zum Vergleich werden über die verfügbare Literatur zusammen getragen und dann die Sprachen mit ihnen ausgewertet.

\section{Gliederung}
Die Arbeit gliedert sich wie folgt in weitere Kapitel. Im Folgenden wird in Kapitel 2 auf die technischen Grundlagen der einzelnen Sprachen, deren Modellkonzeption, Notation und Geschichte eingegangen. Danach konzentriert sich das dritte Kapitel auf die Bewertungskriterien, nach welchen wir die einzelnen Sprachen hinsichtlich ihrer Vor- und Nachteile beschreiben. In Kapitel 4 werden dann die Sprachen hinsichtlich der gennanten Kriterien aus Kapitel 3 analysiert und die Vor- bzw. Nachteile beschrieben.
Das letzte Kapitel 5 handelt von den Erfahrungen und Erkenntnissen, die während
der Arbeit gesammelt wurden. Sie wird noch einmal überblickt und schließt mit einem Fazit ab.
