\chapter{Einleitung}
\label{ch:Einleitung}
Seit einigen Jahrzehnten spielt die Art Prozessabläufe und abstrakte Sachverhalte für den Anwender und Leser in einer möglichst nützlichen Beschreibung darzustellen eine wesentliche Rolle in der Informatik. So ist die Möglichkeit diese verschiedenen Sachverhalte zu beschreiben in den unterschiedlichsten Sprachen, ob formell oder informell, definiert worden. Zwar ist es möglich einen Sachverhalt als Person individuell und intuitiv zu beschreiben und zu Modellieren, jedoch ist die Darstellung des Sachverhalts darauf bedacht, die Verwendung dieser durch mehrere Personen zu ermöglichen. Deswegen ist es notwendig eine einheitliche Syntax zu verwenden und eine Semantik zur Interpretationsminderung zu definieren. So können zum einen Domänenspezifische Sprachen verwendet werden oder globale Modellierungssprachen. Diese genannten Arten von Sprachen dienen zur Abstraktion und zum besseren Verständnis zwischen Mensch zu Mensch bzw. Computer. Sie sind sogenannte Formale Sprachen, welche künstlich erstellte Sprachen sind und sich für eine präzise Beschreibung von Eigenschaften und Verhalten eignen. Sie bestehen aus einer konkreten Syntax, abstrakten Syntax und Semantik. Das vorgestellte Projekt soll nun ausgewählte Modellierungssprachen in Hinsicht auf ihre Funktionsweise und  Kommunikationsabläufe kritisch beobachten, beschreiben und Eignung bzw. Uneignung anhand ausgewählter Kriterien vornehmen. 


%Am Ende soll diese Ausarbeitung eine Aussage über die Eignung von Einsatzzwecken der Sprachen bei Kommunikationsabläufen treffen.

\section{Problemstellung}
\label{sc:Problemstellung}
Da formale Sprachen künstliche Sprachen sind und einen Kontext über eine formale mathematische Syntax beschrieben, werden diese bevorzugt zur Abstraktion von Sachverhalten, wie der von Kommunikationsabläufen in Systemen, verwendet. Nun können unterschiedliche Sprachen einen Sachverhalt unterschiedlich gut oder schlecht wiedergeben. Wobei gut oder schlecht, abhängig des Einsatzortes ist und welche formalen Kriterien eine Wiedergabe erleichtert oder erschwert.

\section{Zielsetzung}
\label{sc:Zielsetzung}
In dieser Projektarbeit gilt es den Einsatz unterschiedlicher Sprachen im Einsatz eines Kommunikationssystems zu beschreiben und zu vergleichen. Der Vergleich ist mit eigens gewählten Bewertungskriterien vorzunehmen und über den Einsatzort der Sprachen in einem Kommunikationssystem zu beschreiben. Dazu sollen die Informationen über die ausgewählten Modellierungssprachen aus den Dokumenten der Spezifikationen \acs{SDL} Z.100 [\ref{SDL_Z.100}], \acs{MSC} Z.120 [\ref{MSC_Z.120}] und OMG \acs{UML} [\ref{UML_2.5.1}] entnommen werden. Die Kriterien zum Vergleich werden über die verfügbare Literatur zusammen getragen und dann die Sprachen mit ihnen ausgewertet.

\section{Gliederung}
\label{sc:Gliederung}
Die Arbeit gliedert sich wie folgt in weitere Kapitel. Im Folgenden wird in Kapitel \ref{ch:Grundlagen} auf die technischen Grundlagen der einzelnen Sprachen, deren Modellkonzeption, Notation und Geschichte eingegangen. Dazu übernimmt Herr Alexander Huber die Verantwortung für Abschnitt \ref{sc:SDL} und Herr Majdi Taher Abschnitte \ref{sc:UML} und \ref{sc:MSC}. Danach konzentriert sich das dritte Kapitel auf die ausgewählten Bewertungskriterien, in denen wir diese beschreiben und den Bezugsraum darstellen. Herr Majdi Taher wird hierzu Abschnitt \ref{sc:FormaleKriterien} und \ref{sc:AnwenderbezogeneKriterien} und Herr Alenxander Huber Abschnitt \ref{sc:AnwendungsbezogeneKriterien} übernehmen. Nach diesen werden wir die einzelnen Sprachen auswerten. In Kapitel \ref{ch:Eignung} werden dann die Sprachen hinsichtlich der genannten Kriterien aus Kapitel \ref{ch:Bewertungskriterien} analysiert und eine Eignung bzw. Uneignung ausgesprochen. Herr Majdi Taher wird hierzu Abschnitt \ref{sc:UMLB} und Herr Alexander Huber Abschnitt \ref{sc:SDLB} und \ref{sc:MSCB} übernehmen.
Das letzte Kapitel \ref{ch:Fazit} handelt von den Erfahrungen und Erkenntnissen, die während
der Arbeit gesammelt wurden. Sie wird noch einmal überblickt und schließt mit einem Fazit ab. Für die Einleitung war Herr Alexander Huber verantwortlich.

