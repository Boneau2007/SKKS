\section{Gliederung}
\label{sc:Gliederung}
Die Arbeit gliedert sich wie folgt in weitere Kapitel. Im Folgenden wird in Kapitel \ref{ch:Grundlagen} auf die technischen Grundlagen der einzelnen Sprachen, deren Modellkonzeption, Notation und Geschichte eingegangen. Dazu übernimmt Herr Alexander Huber die Verantwortung für Abschnitt \ref{sc:SDL} und Herr Majdi Taher Abschnitte \ref{sc:UML} und \ref{sc:MSC}. Danach konzentriert sich das dritte Kapitel auf die ausgewählten Bewertungskriterien, in denen wir diese beschreiben und den Bezugsraum darstellen. Herr Majdi Taher wird hierzu Abschnitt \ref{sc:FormaleKriterien} und \ref{sc:AnwenderbezogeneKriterien} und Herr Alenxander Huber Abschnitt \ref{sc:AnwendungsbezogeneKriterien} übernehmen. Nach diesen werden wir die einzelnen Sprachen auswerten. In Kapitel \ref{ch:Eignung} werden dann die Sprachen hinsichtlich der genannten Kriterien aus Kapitel \ref{ch:Bewertungskriterien} analysiert und eine Eignung bzw. Uneignung ausgesprochen. Herr Majdi Taher wird hierzu Abschnitt \ref{sc:UMLB} und Herr Alexander Huber Abschnitt \ref{sc:SDLB} und \ref{sc:MSCB} übernehmen.
Das letzte Kapitel \ref{ch:Fazit} handelt von den Erfahrungen und Erkenntnissen, die während
der Arbeit gesammelt wurden. Sie wird noch einmal überblickt und schließt mit einem Fazit ab. Für die Einleitung war Herr Alexander Huber verantwortlich.
