\section{Gliederung}
\label{sc:Gliederung}
Die Arbeit gliedert sich wie folgt in weitere Kapitel. Im Folgenden wird in Kapitel \pageref{ch:Grundlagen} auf die technischen Grundlagen der einzelnen Sprachen, deren Modellkonzeption, Notation und Geschichte eingegangen. Dazu übernimmt Herr Alexander Huber die Verantwortung für Abschnitt \pageref{sc:SDL} und Herr Majdi Taher Abschnitte \pageref{sc:UML} und \pageref{sc:MSC}. Danach konzentriert sich das dritte Kapitel auf die ausgewählten Bewertungskriterien, in denen wir diese beschreiben und den Bezugsraum darstellen. Herr Majdi Taher wird hierzu Abschnitt \pageref{sc:FormaleKriterien} und \pageref{sc:AnwenderbezogeneKriterien} und Herr Alenxander Huber Abschnitt \pageref{sc:AnwendungsbezogeneKriterien} übernehmen. Nach diesen werden wir die einzelnen Sprachen auswerten. In Kapitel \pageref{ch:Eignung} werden dann die Sprachen hinsichtlich der genannten Kriterien aus Kapitel \pageref{ch:Bewertungskriterien} analysiert und eine Eignung bzw. Uneignung ausgesprochen. Herr Majdi Taher wird hierzu Abschnitt \pageref{sc:UMLB} und Herr Alexander Huber Abschnitt \pageref{sc:SDLB} und \pageref{sc:MSCB} übernehmen.
Das letzte Kapitel \pageref{ch:Fazit} handelt von den Erfahrungen und Erkenntnissen, die während
der Arbeit gesammelt wurden. Sie wird noch einmal überblickt und schließt mit einem Fazit ab.
