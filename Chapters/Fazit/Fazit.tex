\chapter{Fazit}
\label{ch:fazit}
Eine strukturierte Liste von Beurteilungskriterien lässt sich daraus schwer ableiten.
Dazu sind Evaluationen von Modellierungssprachen ein schwieriges Unterfangen.
Dies liegt zum einen in der Komplexität des Untersuchungsgegenstandes begründet.
Zum anderen ist es problematisch, objektiv bewertbare Qualitätskriterien zu
identifizieren: Qualitätskriterien für Modellierungssprachen machen
sich eben nicht allein an einer formalen Syntax und Semantik oder einer Notation fest.
Stattdessen ist es von entscheidender Bedeutung, das Verhältnis der Sprache zu den
Modellierern und dem Modellierungszweck zu bewerten. Dies bedeutet, dass die Bewertungskriterien
von Fall zu Fall unterschiedlich stark ins Gewicht fallen und dass die
Bewertung der einzelnen Kriterien von Fall zu Fall ebenfalls unterschiedlich ist. Daraus
folgt, dass die Evaluation kaum zu 100 Prozent objektiv sein kann. \\
Dies muss sie aber auch nicht sein: Eine Modellierungssprache muss „lediglich“ – dies
ist aber bereits schwer genug zu erreichen – zu den Modellierern und dem Modellierungszweck
passen. Es schadet für die Praxis nicht, persönliche Präferenzen – und diese
fließen fast zwangsläufig in die praktische Evaluation ein – zu beachten. Den Modellierern
muss die Sprache gefallen, denn sie müssen damit kreativ und sicher umgehen
können.\\
Dieser Bezugsrahmen kann und soll nicht den Anspruch auf Vollständigkeit erheben.
Dennoch stellt er eine geeignete – durchaus durch eigene weitere Kriterien erweiterbare
– Grundlage für ein Evaluationsprojekt dar, denn es findet sich in diesem Bezugsrahmen
eine große Anzahl von Anforderungen an Modellierungssprachen,
die in einer Evaluation Beachtung finden sollten. Die Evaluation sollte jedoch von erfahrenen
Modellierern durchgeführt werden, die dazu in der Lage sind zu bewerten, in
wie fern die Anforderungen erfüllt sind, die die Anforderungen in ihrer Wichtigkeit
bewerten, und gegebenenfalls eine Auswahl treffen, welche Kriterien innerhalb eines
Evaluierungsprojekts zu untersuchen sind, und welchen für einen gegeben Modellierungszweck
und Modellierungsumfang weniger Beachtung geschenkt werden muss.\\
Mit dem Bezugsrahmen ist aber nicht nur ein Beitrag zur Evaluation von Modellierungssprachen
geleistet, sondern auch zur Entwicklung oder Weiterentwicklung derselben. Die Anforderungen können sukzessive durchgearbeitet und geeignete Konzepte in die Modellierungssprache aufgenommen werden: Dieser Bezugsrahmen soll damit durchaus auch die Kreativität der Entwickler von Modellierungssprachen fördern.

