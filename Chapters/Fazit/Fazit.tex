\chapter{Fazit}
\label{ch:Fazit}
Es hat sich herausgestellt, das die Schaffung, Ordnung und Selektion von Bewertungskriterien ein schwieriges unterfangen sein kann.
So lassen sich zunächst nur schwer Begrifflichkeiten fest machen, um eine Liste für seinen Untersuchungspunkt fest zu machen.
Genauso ergeht es geeignete Modellierungssprachen dahingehend zu evaluieren, was der Komplexität des Unterfangens geschuldet ist.
Die Problematik besteht unter anderem darin, objektiv bewertbare Qualitätskriterien zu identifizieren,
da diese sich nicht alleine an einer formalen Syntax und Semantik oder Notation fest zu machen sind. Es erschließt sich eher daher, in welchem Verhältnis die Modellierungssprache zu einem Anwendungszweck steht und wie dieser zu bewerten ist. So variiert die Gewichtung von Bewertungskriterium von Anwendungszweck zu Anwendungszweck, sodass jede Bewertung von Fall zu Fall einzeln betrachtet werden muss. Demnach kann von einer zuverlässigen objektiven Betrachtung nicht die rede sein. Jedoch ist nur von entscheidender Bedeutung, das die Modellierungssprache zum Anwendungszweck passt. Persönliche Erfahrungen mit unterschiedlichen Modellierungssprachen in unterschiedlichen Anwendungsfeldern können einem bei der Evaluation zwar nützlich sein, jedoch fließt dadurch zwangsläufig auch eine persönliche Präferenz in sie mit ein. Bei wenig bis gar keiner Kenntnis kann es jedoch auch passieren, das Punkte der Betrachtung ausgelassen werden, Konzepte und Konstrukte falsch interpretiert werden könnten. Unser Bezug soll nicht Ansprüche auf Vollständigkeit stellen. Dennoch soll er dem Betrachter einen ersten kritischen Einblick in die Auswahl seiner Modellierungssprache gewähren. Dazu können die Angewandten Bewertungskriterien auch auf beliebig andere Modellierungssprachen angewendet werden um einen erweiterten Vergleich zu vereinfachen. Da wir keine erfahrenen Modellierer in all unseren gewählten Sprachen waren, konnten wir möglicherweise nicht die Wichtigkeit der einzelnen Anforderungen oder Kriterien einzeln bewerten, aber ein grobes Verständnis für den jeweiligen Modellierungszweck vermitteln. 
