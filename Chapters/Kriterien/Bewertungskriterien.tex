

\chapter{Bewertungskriterien}
\label{ch:Bewertungskriterien}
In diesem Kapitel werden die für eine Eignung bzw. Uneignung benötigten Kriterien vorgestellt.
Sie sollen helfen, eine Modellierungssprache zu bewerten. Dazu sind die Kriterien in Hauptkategorien untergliedert und werden im Verlauf weiter aufgelöst. 
Dabei handelt es sich um Eigenschaften und deren Ausprägung in einer Modellierungssprache.
Sie können auf die Modellierungssprache in ihrer Gesamtheit, aber auch auf einzelne Bestandteile oder Sprachmittel angewendet werden.
In vielen Fällen bestehen zwischen den einzelnen Eigenschaften enge Beziehungen.
Je nachdem für welchen Sprachbestandteil die Bewertung vorgenommen wird, können andere Maße für die Bewertung verwendet werden.
Für viele der hier vorgestellten Merkmale gibt es jedoch keine exakten Bewertungsverfahren, inwieweit das Merkmal erfüllt ist oder nicht.
Die Kriterien sind in drei Kategorien eingeteilt, auf welche im Folgenden eingegangen wird:

\begin{itemize}
	\item Formale Kriterien
	\item Anwenderbezogene Kriterien
	\item Anwendungsbezogene Kriterien
\end{itemize}


\section{Formale Kriterien}  
Diese Kriterien dienen der maschinellen Prüfung von Modellen sowie der Berechnung von Modelleigenschaften. Werden die Anforderungen von der Modellierungssprache erfüllt, kann ein mit ihr geschaffenes Prozessmodell z. B. auf syntaktische Korrektheit überprüft werden. Die formalen Kriterien spielen eine besondere Rolle, wenn für die Modellierung Softwarewerkzeuge eingesetzt werden.\cite{MT007}\\
Die formalen Kriterien werden in die folgende Einzelkriterien untergeteilt:  Korrektheit, Vollständigkeit, Einheitlichkeit, Redundanzfreiheit, Strukturierbarkeit, Wiederverwendbarkeit und Wartbarkeit.
\subsection{Korrektheit}
Das Kriterium der Korrektheit steht in Verbindung zum Grundsatz der Richtigkeit der Grundsätze ordnungsgemäßer Modellierung.
Das Merkmal hier erfüllt, wenn man einfach die unkorrekte Modelle identifizieren kann.
Bei diesem Kriterium gibt es zwei unterschiedliche Ausprägung der Korrektheit und sie sind: syntaktische- sowie die semantische -eindeutige Identifikation fehlerhafter Modelle. 
Die syntaktische fehlerhafte Modelle sind eindeutig erkennbar ,d.h. die nicht eindeutig identifizierbar fehlerhafte Modelle sind korrekt. Der gleiche Fall ist bei der semantischen eindeutigen Identifikation fehlerhafter Modelle, wobei sind die semantische fehlerhafter Modelle eindeutig erkennbar, wenn die nicht eindeutig identifizierbar fehlerhafte Modelle richtig sind.
\subsection{Vollständigkeit}
Es geht hier um die Vollständigkeit der Sprachbeschreibung. Das Merkmal erfüllt, wenn alle benötigte Modelle mit dieser Modellierungssprache modellierbar sind.
Ein vollständiges Modell kann erstellt werden, wenn alle Modelle und die Bedingungen ihrer Verwendung eindeutig und komplett definiert können.
\subsection{Einheitlichkeit}
Unter Vollständigkeit wird in diesem Zusammenhang die ähnliche Darstellung von den ähnlichen Konzepte verstanden.
Um das Merkmal der Einheitlichkeit zu erfüllen, muss alle Konstrukte der Sprache verständlich dargestellt und beschrieben werden.
\subsection{Redundanzfreiheit}
Die Redundanzfreiheit setzt voraus, dass die Sprache keine Mehrfachdefinition von Konstrukten vornimmt, die denselben Sachverhalt beschreiben. Ein und derselbe Sachverhalt sollte also nicht mit mehreren verschiedenen Elementen bzw. Symbolen der Modellierungssprache belegt sein.\cite{MT007}\\
Der Redundanzfreiheit bedeutet, dass es keine redundante Informationen im Modell gibt. Die unnötige redundante Informationen können der Benutzer ablenken und sie hilfen auf keinen Fall in der Modellbeschreibung, sonst sie bilden mehr Komplexität im Modell und man soll dafür mehr Zeit investieren, um dieses Modell zu verstehen.
Nur wenn im Modele die Redundanzen vermieden werden, kann ein
redundanzfreie Modell erstellt werde.
\subsection{Strukturierbarkeit, Wiederverwendbarkeit und Wartbarkeit}
das letzte Teil von den formalen Kriterien ist die Strukturierbarkeit, Wiederverwendbarkeit und Wartbarkeit, die miteinander verbunden sind. Dabei sollte die Modellierungssprache Konstrukte bereitstellen, welche die Strukturierung der modellierten Informationen unterstützt.\\
Die Modellierungssprache muss also Zerlegungen in Prozesskomponenten oder Teilprozesse darstellen können. Die Schnittstellen zwischen den Komponenten müssen übersichtlich dargestellt werden.\cite{MT002}
Um das Merkmal zu erfüllen, müssen die Generalisierungen und die Spezialisierung schlüssig nachvollziehbar sein.\\
Die Strukturierung in Teilmodelle ermöglicht gleichzeitig die Wiederverwendbarkeit der Strukturen in anderen Modellen. So können etwa modellierte Teilprozesse in anderen Prozessen wieder aufgegriffen werden und müssen nicht mehrfach modelliert werden.\cite{MT007}Dazu ist die Wartbarkeit der Modelle durch die Strukturierung verbessert, weil die Modifikation des Modellteils verändert nicht die Konsistenz der übrigen Modellteile.
Wie bei der Einheitlichkeit fördert das Kriterium der Strukturierbarkeit die Grundsätze ordnungsgemäßer Modellierung.
\section{Anwenderbezogene Kriterien}
Die anwenderbezogenen Kriterien beschreiben das Verhältnis zwischen dem Anwender und der verwendeten Modellierungssprache. Der Anwender kann zum einen der Modellierer und zum anderen der Betrachter eines Modells sein. Die anwenderbezogenen Kriterien besitzen eine besondere Bedeutung, um Nutzern von Modellen das Verständnis zu erleichtern bzw. zu ermöglichen. Für den betrachteten Anwendungsfall der Prozessdokumentation, zur Nutzung als allgemeine Arbeitsgrundlage für die Mitarbeiter einer Organisation, spielen sie daher eine sehr große Rolle. Die Modellierungssprache ermöglicht das Erstellen von Modellen, um Ideen und Gedanken zwischen den Beteiligten auszutauschen. Sie sind somit in erster Linie für den menschlichen Anwender als Kommunikationsmittel zu verstehen.\cite{MT007}   
\subsection{Einfachheit und Erlernbarkeit}
Das Kriterium der Einfachheit geht von folgendem Prinzip aus: Je einfacher eine Sprache ist, desto weniger Fehler sind bei der Modellierung mit ihr zu erwarten. Die Einfachheit der Modellierungssprache wird bestimmt durch eine geringe Anzahl an Notationselementen und Begriffen, sowie der Verwendung von einfachen Regeln für ihre Anwendung. Der Aspekt betrifft zum einen den Ersteller eines Modells, der bei einfachen Sprachen nur Kenntnis von wenigen Symbolen und Regeln haben muss, um Modelle zu erstellen. Wichtiger sind bei der Prozessdokumentation jedoch die Betrachter der Modelle. Dazu zählen nicht nur Fachexperten, sondern häufig einfache Arbeiter in den Unternehmen. Diese besitzen meist nur sehr eingeschränkte bis gar keine Methodenkenntnisse auf dem Gebiet der Prozessmodellierung. Trotzdem müssen die Modelle verstanden werden, wenn sie als Arbeitsgrundlage Verwendung finden sollen. \\
Eine Sprache, die für die Dokumentation von Prozessen verwendet wird, sollte nicht nur möglichst einfach sein, sondern auch den erforderlichen Schulungsbedarf auf einem akzeptablen Niveau halten. Vom Kriterium der Erlernbarkeit wird also ein möglichst geringer Aufwand zum Erlernen der notwendigen Konstrukte und Regeln der Modellierungssprache gefordert. Dies hängt wiederum stark von der Anzahl der Konstrukte der Sprache ab, sodass die Einfachheit und die Erlernbarkeit sehr stark korrelieren. Deshalb wurden sie in einem gemeinsamen Kriterium zusammengeführt. Das Kriterium der Einfachheit und Erlernbarkeit unterstützt den Grundsatz der Klarheit der GoM. Außerdem wird der Grundsatz der Wirtschaftlichkeit angesprochen, da insbesondere der Schulungsaufwand zum Nachvollziehen der Modellierungssprache und die Ausbildung zur Geschäftsprozessmodellierung in der Organisation einen erheblichen Kostenfaktor darstellen können.
\subsection{Verständlichkeit}
Die Modellierungssprache gilt als verständlich, wenn sie ein dem Nutzer bekanntes Vokabular benutzt, d. h. die definierten Konstrukte sollten im Wesentlichen mit Begriffen korrespondieren, die dem Anwender vertraut sind. Auf den Zweck der Prozessdokumentation übertragen bedeutet dies, dass die Symbole der Notation eher mit betriebswirtschaftlichen und allgemein bekannten Begriffen besetzt sein sollten, als etwa mit Begriffen aus der Softwareentwicklung oder anderen technischen Richtungen. Die Begriffe müssen dem Anwender aus seiner täglichen Arbeit bekannt sein, damit die Bedeutung leicht zu interpretieren ist. Auch die Verständlichkeit einer Modellierungssprache fördert die Qualität des damit abgebildeten Ergebnismodells und unterstützt somit den Grundsatz der Klarheit.

\subsection{Anschaulichkeit}
Während die Verständlichkeit eher auf die Benennung und verständliche Definition der Konstrukte ausgerichtet ist, umfasst das Kriterium der Anschaulichkeit die Forderung nach Struktur, Übersichtlichkeit und Lesbarkeit der Modelle, um den Grundsatz der Klarheit zu fördern. Die grafische Darstellung sollte möglichst intuitive Konzepte einsetzen um die Anschaulichkeit zu erhöhen. Dies kann z. B. die Verwendung von Piktogrammen sein, die realweltlichen Objekten nachempfunden sind. Die Lesbarkeit sinkt, umso mehr verschiedene Notationselemente verwendet werden. Es wird davon ausgegangen, dass der Mensch durch das direkte Betrachten nur etwa sechs verschiedene Notationselemente unterscheiden kann. Die Struktur beschreibt, wie die Elemente in einem Modell gruppiert werden. So können z. B. Überschneidungen von Kanten die Anschaulichkeit eines Modells erheblich reduzieren.



\section{Anwendungsbezogene Kriterien}
\label{sc:AnwendungsbezogeneKriterien}
Anwendungsbezogene Kriterien beschreiben die Anforderungen einer Modellierungssprache innerhalb eines Domänenspezifischen Einsatzes.
Dies hängt von den jeweiligen Aspekten ab, welche die Modellierungssprache abbilden soll.
So besitzen verschiedene Modellierungssprachen unter anderem ein unterschiedlich großes Nutzungspotenzial in der jeweiligen Anwendungsdomäne.
Man spricht in diesem Zusammenhang auch von der Mächtigkeit der Sprache.
Dabei ist zu beachten, das Anwendungs- und Anwenderbezogene Kriterien oft konkurrierend sein können,
so kann sich beispielsweise eine leicht erlernbare Sprache sich negativ auf die Mächtigkeit auswirken und umgekehrt.
Anwendungsbezogene Kriterien können in Anforderungen wie der Zielsetzung, der Mächtigkeit, der Operationalisierbarkeit,
dem Abstraktionsniveau, dem Detaillierungsgrad und dem Formalisierungsgrad beschrieben werden.
Zwar gibt es höchstwahrscheinlich noch eine weit aus größere Anzahl an Kriterien, diese sollen uns aber in dieser Arbeit genügen.  
[Ein Konzept zur Simulation wissensintensiver Aktivitäten in Geschäftsprozessen 95F]

\subsection{Zielsetzung}
\label{ssc:Dokumentation}
Die Zielsetzung beschreibt den wesentlichen Zweck mit dem eine Modellierungssprache für eine Anwendungsdomäne einer Modellierungssprache wird vornehmlich in dessen Dokumentation erläutert. In ihr wird der wesentliche Zweck der Sprache beschrieben.

Ist die Dokumetation Vollständig? Und warum ist die Vollständigkeit wichtig?
Hat die Dokumentation einen einfach zu versehenden Aufbau
Ist die Dokumentation Verständlich geschrieben.
Sind die Dokumente Aktuell?

\subsection{Mächtigkeit}
\label{ssc:Nutzungspotenzial}
Die Mächtigkeit ist ein maß des Nutzungspotenzials einer Sprache und gibt Aussage darüber,
in wie weit und wie gut die Konzepte der verwendeten Sprache, die Eigenschaften eines Sachverhalt der Domäne darstellen kann.
Darunter fällt wie Präzise diese Aussagen sind und wie hoch der Detailgrad der darzustellenden Eigenschaft ist werden [Allweyer 2005b, S.180].
Der Sprachumfang korreliert mit der Mächtigkeit der Sprache. Je größer dieser Umfang ist, desto größer ist auch die Mächtigkeit der Sprache.
Da es für den Anwender von großer Bedeutung ist, seine Anwendung mit einem möglichst umfänglichen Detailgrad beschreiben zu können,
muss die Mächtigkeit mindestens alle Aspekte enthalten, die für den gewünschten darzustellenden Sachverhalt notwendig sind.
Eine Mächtigkeit der Sprache, die sich über den Anwendungszweck der Domäne bezieht, kann wie schon erwähnt sich negativ auf andere Kriterien auswirken.
Da alle Modellierungssprachen formale Sprachen sind, kann man die Mächtigkeit ihrer Sprache anhand ihrer Grammatik-Typen festlegen.

\subsection{Operationalisierbarkeit}
\label{ssc:Operationalisierbarkeit}
Die Operationalisierbarkeit gibt Aussage darüber, ob und wie gut sich die Modellierungssprache über ihre eigene Verwendung hinaus noch weiter verwenden lässt.
Darunter fällt die Transformation des Modells in andere Sprachen als auch das darstellen von diversen Sachverhalten.
Bei der Transformation ist hierbei zu beachten, dass die verwendeten Konzepte beider Sprachen möglichst gleich sein müssen,
um eine annähernd vollständige Konvertierung zu ermöglichen. Zur Abbildung diverser Sachverhalte enthält eine gute Operationalisierbarkeit die Abbildungsmöglichkeit von Funktionen, Bedingungen, Ressourcen, Objekten und Ereignissen.
Deswegen müssen Konzepte zur Planung und Analyse des domänenspezifischen Einsatzes enthalten sein.
[IT-Landschaften 25]

\subsection{Abstraktionsniveau}
\label{ssc:Abstraktionsniveau}
Die Abstraktionsfähigkeit einer Modellierungssprache beschreibt ein Kriterium, worin die Fachterminologie der Domäne sich mit den Konzepten der Modellierungssprache möglichst decken sollte. In dem Bereich von Kommunikationsabläufen der Telekommunikation sind diese Begriffe eher technischer Natur und kommen aus einem Informatik-spezifischem Umfeld, welcher Fachsprache auf dem Niveau von Spezialisten voraussetzt. 

\subsection{Detailgrad}
\label{ssc:Detailgrad}
Der Detailgrad beschreibt wie Sachlich angemessen detailliert die Konstrukte eines Anwendungszwecks der Domäne sich darstellen lassen können. Dies beinhaltet die zu Modellierenden Modelle, Daten und zusätzlichen Informationen. Damit ist nicht gemeint, dass ein System in einer hohen Abstraktionsform eine eins zu eins Nachbildung 
aller technischen Prozesse des Informationssystems ausdrücken soll, sondern nur für die Zielgruppe angemessene und verständliche Konzepte. Eine formale Beschreibung der Prozesse ist demnach nicht gewünscht, da der Hauptaugenmerk auf dem Anwender liegt.

\subsection{Funktionalität}
\label{ssc:Funktionalität}
Anforderungen
Anpassung
Flexibilität



