\section{Anwenderbezogene Kriterien}
\label{sc:AnwenderbezogeneKriterien}
Die anwenderbezogenen Kriterien beschreiben das Verhältnis zwischen dem Anwender und der verwendeten Modellierungssprache. Der Anwender kann zum einen der Modellierer und zum anderen der Betrachter eines Modells sein. Die anwenderbezogenen Kriterien besitzen eine besondere Bedeutung, um Nutzern von Modellen das Verständnis zu erleichtern bzw. zu ermöglichen. Für den betrachteten Anwendungsfall der Prozessdokumentation, zur Nutzung als allgemeine Arbeitsgrundlage für die Mitarbeiter einer Organisation, spielen sie daher eine sehr große Rolle. Die Modellierungssprache ermöglicht das Erstellen von Modellen, um Ideen und Gedanken zwischen den Beteiligten auszutauschen. Sie sind somit in erster Linie für den menschlichen Anwender als Kommunikationsmittel zu verstehen.\cite{MT007}   
\subsection{Einfachheit und Erlernbarkeit}
Das Kriterium der Einfachheit geht von folgendem Prinzip aus: Je einfacher eine Sprache ist, desto weniger Fehler sind bei der Modellierung mit ihr zu erwarten. Die Einfachheit der Modellierungssprache wird bestimmt durch eine geringe Anzahl an Notationselementen und Begriffen, sowie der Verwendung von einfachen Regeln für ihre Anwendung. Der Aspekt betrifft zum einen den Ersteller eines Modells, der bei einfachen Sprachen nur Kenntnis von wenigen Symbolen und Regeln haben muss, um Modelle zu erstellen. Wichtiger sind bei der Prozessdokumentation jedoch die Betrachter der Modelle. Dazu zählen nicht nur Fachexperten, sondern häufig einfache Arbeiter in den Unternehmen. Diese besitzen meist nur sehr eingeschränkte bis gar keine Methodenkenntnisse auf dem Gebiet der Prozessmodellierung. Trotzdem müssen die Modelle verstanden werden, wenn sie als Arbeitsgrundlage Verwendung finden sollen. \\
Eine Sprache, die für die Dokumentation von Prozessen verwendet wird, sollte nicht nur möglichst einfach sein, sondern auch den erforderlichen Schulungsbedarf auf einem akzeptablen Niveau halten. Vom Kriterium der Erlernbarkeit wird also ein möglichst geringer Aufwand zum Erlernen der notwendigen Konstrukte und Regeln der Modellierungssprache gefordert. Dies hängt wiederum stark von der Anzahl der Konstrukte der Sprache ab, sodass die Einfachheit und die Erlernbarkeit sehr stark korrelieren. Deshalb wurden sie in einem gemeinsamen Kriterium zusammengeführt. Das Kriterium der Einfachheit und Erlernbarkeit unterstützt den Grundsatz der Klarheit der GoM. Außerdem wird der Grundsatz der Wirtschaftlichkeit angesprochen, da insbesondere der Schulungsaufwand zum Nachvollziehen der Modellierungssprache und die Ausbildung zur Geschäftsprozessmodellierung in der Organisation einen erheblichen Kostenfaktor darstellen können.
\subsection{Verständlichkeit}
Die Modellierungssprache gilt als verständlich, wenn sie ein dem Nutzer bekanntes Vokabular benutzt, d. h. die definierten Konstrukte sollten im Wesentlichen mit Begriffen korrespondieren, die dem Anwender vertraut sind. Auf den Zweck der Prozessdokumentation übertragen bedeutet dies, dass die Symbole der Notation eher mit betriebswirtschaftlichen und allgemein bekannten Begriffen besetzt sein sollten, als etwa mit Begriffen aus der Softwareentwicklung oder anderen technischen Richtungen. Die Begriffe müssen dem Anwender aus seiner täglichen Arbeit bekannt sein, damit die Bedeutung leicht zu interpretieren ist. Auch die Verständlichkeit einer Modellierungssprache fördert die Qualität des damit abgebildeten Ergebnismodells und unterstützt somit den Grundsatz der Klarheit.

\subsection{Anschaulichkeit}
Während die Verständlichkeit eher auf die Benennung und verständliche Definition der Konstrukte ausgerichtet ist, umfasst das Kriterium der Anschaulichkeit die Forderung nach Struktur, Übersichtlichkeit und Lesbarkeit der Modelle, um den Grundsatz der Klarheit zu fördern. Die grafische Darstellung sollte möglichst intuitive Konzepte einsetzen um die Anschaulichkeit zu erhöhen. Dies kann z. B. die Verwendung von Piktogrammen sein, die real-weltlichen Objekten nachempfunden sind. Die Lesbarkeit sinkt, umso mehr verschiedene Notationselemente verwendet werden. Es wird davon ausgegangen, dass der Mensch durch das direkte Betrachten nur etwa sechs verschiedene Notationselemente unterscheiden kann. Die Struktur beschreibt, wie die Elemente in einem Modell gruppiert werden. So können z. B. Überschneidungen von Kanten die Anschaulichkeit eines Modells erheblich reduzieren.


