\section{Anwenderbezogene Kriterien}
\label{sc:AnwenderbezogeneKriterien}
Man benutzt die Modellierungssprachen, um Ideen, Vorstellungen und Gedanken in einem Model zu Beschreiben, deshalb sind die Modellierungssprachen hauptsächlich an die Menschen gerichtet.
Das Verhältnis zwischen dem Anwender und der Modellierungssprachen ist durch die anwenderbezogenen Kriterien beschrieben.\\
Aus dieser Gründen sind die anwenderbezogenen Kriterien sehr wichtig und spielen eine große Rolle bei der Eignungsbewertung der Modellierungssprachen.   
\subsection{Einfachheit und Erlernbarkeit}
\label{sc:Einfachheit}
Das Kriterium der Einfachheit geht von folgendem Prinzip aus: Je einfacher eine Sprache ist, desto weniger Fehler sind bei der Modellierung mit ihr zu erwarten. Die Einfachheit der Modellierungssprache wird bestimmt durch eine geringe Anzahl an Notationselementen und Begriffen, sowie der Verwendung von einfachen Regeln für ihre Anwendung\cite{MT014}.Dazu geht auch das Kriterium der Einfachheit von diesem Prinzip aus: Je weniger Regeln zu beachten sind, desto weniger Fehler können in den Modellen vorhanden sein. Dadurch
wird die Sicherheit bei der Modellierung gesteigert\cite{MT006}.
Um das Merkmal bon Einfachheit zu erfüllen, muss man keine tiefe Kenntnisse von Symbolen und Regeln haben, um ein Modell mit einer Modellierungssprache zu erstellen. Auf jeden Fall ist bei jeder Modellierungssprache ein Grundkenntnisse von den Regeln, die Notation und die Symbolen benötigt, aber diese Grundkenntnisse muss schnell gelernt werden und benötigt keine Ausbildung oder bestimmte Voraussetzungen. Man soll das größte Teil der Zeit in dem Entwurf selbst investieren und nicht im Lernen einer Modellierungssprache. \\
Wichtiger sind bei der Prozessdokumentation jedoch die Betrachter der Modelle. Dazu zählen nicht nur Fachexperten, sondern häufig einfache Arbeiter in den Unternehmen. Diese besitzen meist nur sehr eingeschränkte bis gar keine Methodenkenntnisse auf dem Gebiet der Prozessmodellierung\cite{Lobe_2015}.



\subsection{Verständlichkeit}
\label{sc:Verständlichkeit}
Unter Verständlichkeit wird in diesem Zusammenhang die Verständlichkeit und die Leichtigkeit der Sprachbeschreibung verstanden. Hier muss die Modellierungssprachen mit einfache Fachterminologie und Begriffen und bekannte Symbolen, die dem Anwender vertraut sind, beschrieben werden. Damit sind diese Fachterminologie für den Anwender verstehbar. \\
Die Begriffe müssen dem Anwender aus seiner täglichen Arbeit bekannt sein, damit die Bedeutung leicht zu interpretieren ist\cite{MT014}.
Gleichzeitig muss die Moddielierungssprachen mit zielorientierte n 
Begriffen beschrieben werden, um die Idee des Anwenders richtig und eindeutig zu beschreiben.\\
Je direkter die in der Modellierungssprache verwendeten
Konzepte mit den dem Anwender bekannten Begriffen
korrespondieren, desto leichter können die Modelle
vom fachlich qualifizierten Anwender erstellt
und verstanden werden\cite{MT006}.

\subsection{Anschaulichkeit}
\label{sc:Anschaulichkeit}
Das Kriterium der Anschaulichkeit umfasst die Forderung nach Struktur, Übersichtlichkeit und Lesbarkeit der Modelle, um den Grundsatz der Klarheit zu fördern. Die grafische Darstellung sollte möglichst intuitive Konzepte einsetzen um die Anschaulichkeit zu erhöhen. Dies kann z. B. die Verwendung von Piktogrammen sein, die realweltlichen Objekten nachempfunden sind\cite{Lobe_2015}.
Hier geht es einerseits um eine dem Modellbegriff schon fast inhärente Forderung: Ein Modell sollte
anschaulich sein. Andererseits ist hier an den Umstand zu denken, dass Modelle der Abbildung faktischer
oder geplanter Realität dienen. Solche Abbildungen können grundsätzlich fehlbar sein. Ein
Modell sollte seine Überprüfung an der Realität unterstützen.
Ein Modell ist dann anschaulich, wenn es möglichst direkt mit Wahrnehmungsmustern und Konzeptualisierungen
des Betrachters korrespondiert. Wahrnehmungsmuster und Konzeptualisierungen
streuen aber bekanntlich interpersonell und intrapersonell (der Mensch ist lernfähig). Anschaulichkeit
in diesem Sinn hängt einerseits den jeweils gewählten Abstraktionen und Bezeichnern ab, andererseits
von der jeweiligen Modellierungssprache (worauf noch einzugehen sein wird). Die Beurteilung der
Beziehung zwischen Modell und Realität liegt i.d.R. allein bei den Betrachtern. Die Anschaulichkeit
eines Modells ist also grundsätzlich geeignet, seine Überprüfbarkeit zu fördern. Dabei ist es wesentlich,
daß das Verhältnis des Modells zur Realität durch möglichst genaue Abbildungsvorschriften
erläutert wird.\cite{MT010}

