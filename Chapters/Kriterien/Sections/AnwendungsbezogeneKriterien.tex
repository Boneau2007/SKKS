\section{Anwendungsbezogene Kriterien}
\label{sc:AnwendungsbezogeneKriterien}
Anwendungsbezogene Kriterien beschreiben die Nutzung von Modellierungssprachen aus Sicht des Anwenders idR. abhängig 
von ihrem Domänenspezifischen Einsatz. So besitzen verschiedene Modellierungssprachen einen unterschiedlich großes 
Nutzungspotenzial im jeweiligen Bereich. Man spricht in diesem Zusammenhang auch von der Mächtigkeit der Sprache. 
Jedoch ist zu beachten, das Anwendungsbezogene Kriterien und Anwenderbezogene Kriterien konkurrieren können, 
da sich beispielsweise eine leicht erlernbare Sprache sich negativ auf die Mächtigkeit auswirken könnte und umgekehrt.

\subsection{Nutzungspotenzial}
\label{ssc:Nutzungspotenzial}
Das Nutzungspotenzial gibt Aussagen darüber, wie gut die verwendete Sprache Eigenschaften in einem darzustellenden Sachverhalt
beschreibt. Darunter fällt wie Präzise diese Aussagen sind und wie hoch der Detailgrad der darzustellenden Eigenschaft ist. 
Je größer dieser Sprachumfang ist, desto größer ist auch die Mächtigkeit der Sprache. Nun ist es schwierig eine Aussage darüber 
zu treffen, inwiefern eine Sprache für eine Anwendungsdomäne, wie der von Telekommunikationsprozessen, eine vollständige
Beschreibung der Konzepte stellt.

\subsection{Funktionalität}
\label{ssc:Funktionalität}
