\section{Anwendungsbezogene Kriterien}
\label{sc:AnwendungsbezogeneKriterien}
Anwendungsbezogene Kriterien beschreiben die Anforderungen einer Modellierungssprache innerhalb eines Domänenspezifischen Einsatzes.
Dies hängt von den jeweiligen Aspekten ab, welche die Modellierungssprache abbilden soll.
So besitzen verschiedene Modellierungssprachen unter anderem ein unterschiedlich großes Nutzungspotenzial in der jeweiligen Anwendungsdomäne.
Man spricht in diesem Zusammenhang auch von der Mächtigkeit der Sprache.
Dabei ist zu beachten, das Anwendungs- und Anwenderbezogene Kriterien oft konkurrierend sein können,
so kann sich beispielsweise eine leicht erlernbare Sprache sich negativ auf die Mächtigkeit auswirken und umgekehrt.
Anwendungsbezogene Kriterien können in Anforderungen wie der Zielsetzung, der Mächtigkeit, der Operationalisierbarkeit,
dem Abstraktionsniveau, dem Detaillierungsgrad und dem Formalisierungsgrad beschrieben werden.
Zwar gibt es höchstwahrscheinlich noch eine weit aus größere Anzahl an Kriterien, diese sollen uns aber in dieser Arbeit genügen.  
[Ein Konzept zur Simulation wissensintensiver Aktivitäten in Geschäftsprozessen 95F]

\subsection{Zielsetzung}
\label{ssc:Dokumentation}
Die Zielsetzung beschreibt den wesentlichen Zweck mit dem eine Modellierungssprache für eine Anwendungsdomäne einer Modellierungssprache wird vornehmlich in dessen Dokumentation erläutert. In ihr wird der wesentliche Zweck der Sprache beschrieben.

Ist die Dokumetation Vollständig? Und warum ist die Vollständigkeit wichtig?
Hat die Dokumentation einen einfach zu versehenden Aufbau
Ist die Dokumentation Verständlich geschrieben.
Sind die Dokumente Aktuell?


\subsection{Angemessenheit}
Das Kriterium der Angemessenheit steht in direktem Zusammenhang mit der Mächtigkeit und dem
Anwendungszweck einer Sprache. Die Angemessenheit ist immer relativ zum Anwendungszweck zu
sehen. Sie bezieht sich sowohl auf die Sprache in ihrer Gesamtheit als auch auf die einzelnen Sprachkonzepte.
Angemessenheit und Mächtigkeit bieten zusammen die Möglichkeit, zu untersuchen, ob
eine Sprache für den Anwendungszweck ausreichende Konzepte bereitstellt, gleichzeitig aber keine
überflüssigen Konzepte enthält.
\subsubsection{Konsequenzen für Modellierungssprachen}
Insbesondere bei sehr speziellen Sprachkonstrukten muß untersucht werden, ob diese Konstrukte nötig
sind oder ob sie nur zu einer Erhöhung der Sprachkomplexität führen. Die Entscheidung, ob ein
Sprachmittel angemessen ist oder nicht, ist jedoch häufig nicht objektiv vornehmbar, sondern hängt
stark von den Präferenzen der jeweiligen Betrachter ab. Im allgemeinen gibt es immer Vor- und Nachteile
für einzelne Sprachmittel, so dass ein sorgsames Abwägen notwendig ist.
\subsubsection{Feststellbarkeit/Meßbarkeit der Beurteilungskriterien} 
Generelle formale Maße zur Bestimmung der Angemessenheit von Modellierungssprachen gibt es
nicht. Die Untersuchung der Angemessenheit eines Sprachelementes kann daher nur subjektiv erfolgen.
Dabei ist zu untersuchen, ob ein Sprachelement bedeutsam für den Anwendungszweck ist oder
nicht.
\subsection{Mächtigkeit}
\label{ssc:Nutzungspotenzial}
Die Mächtigkeit ist ein maß des Nutzungspotenzials einer Sprache und gibt Aussage darüber,
in wie weit und wie gut die Konzepte der verwendeten Sprache, die Eigenschaften eines Sachverhalt der Domäne darstellen kann.
Darunter fällt wie Präzise diese Aussagen sind und wie hoch der Detailgrad der darzustellenden Eigenschaft ist werden [Allweyer 2005b, S.180].
Der Sprachumfang korreliert mit der Mächtigkeit der Sprache. Je größer dieser Umfang ist, desto größer ist auch die Mächtigkeit der Sprache.
Da es für den Anwender von großer Bedeutung ist, seine Anwendung mit einem möglichst umfänglichen Detailgrad beschreiben zu können,
muss die Mächtigkeit mindestens alle Aspekte enthalten, die für den gewünschten darzustellenden Sachverhalt notwendig sind.
Eine Mächtigkeit der Sprache, die sich über den Anwendungszweck der Domäne bezieht, kann wie schon erwähnt sich negativ auf andere Kriterien auswirken.
Da alle Modellierungssprachen formale Sprachen sind, kann man die Mächtigkeit ihrer Sprache anhand ihrer Grammatik-Typen festlegen. \\
Eine Modellierungssprache sollte sich auf die Modellierung konzentrieren.
So können zu viele Konzepte zu einer unangemessen mächtigen Sprache führen. Zwischen Mächtigkeit
und Angemessenheit muss daher ein ausgewogenes Verhältnis existieren.\\
Mächtigkeit kann aber auch relativ zur Mächtigkeit von Turingmaschinen betrachtet werden und ist
dann ein Maß für die Berechnungsfähigkeit einer Modellierungssprache. Für Modellierungssprachen
ist diese Betrachtung jedoch nicht ganz korrekt, da sie im allgemeinen keine Formalisierungen von
Algorithmen sind.
\subsubsection{Konsequenzen}
Mächtigkeit und Angemessenheit bedingen einander. Die Mächtigkeit einer Sprache muss daher relativ
zum Anwendungsbereich betrachtet werden. Geht man grundsätzlich davon aus, dass die hier betrachteten
Modellierungssprachen der konzeptuellen Modellierung dienen, misst sich die Mächtigkeit einerseits
an der Möglichkeit, als wesentlich erkannte Sachverhalte des abzubildenden Realitätsbereichs
natürlich, also ohne aufwendige Rekonstruktionen, beschreiben zu können. Andererseits sollte die
Modellierungssprache auch Möglichkeiten bieten, Informationen darzustellen, die für die Implementierung
benötigt werden.

\subsection{Operationalisierbarkeit}
\label{ssc:Operationalisierbarkeit}
Die Operationalisierbarkeit gibt Aussage darüber, ob und wie gut sich die Modellierungssprache über ihre eigene Verwendung hinaus noch weiter verwenden lässt.
Darunter fällt die Transformation des Modells in andere Sprachen als auch das darstellen von diversen Sachverhalten.
Bei der Transformation ist hierbei zu beachten, dass die verwendeten Konzepte beider Sprachen möglichst gleich sein müssen,
um eine annähernd vollständige Konvertierung zu ermöglichen. Zur Abbildung diverser Sachverhalte enthält eine gute Operationalisierbarkeit die Abbildungsmöglichkeit von Funktionen, Bedingungen, Ressourcen, Objekten und Ereignissen.
Deswegen müssen Konzepte zur Planung und Analyse des domänenspezifischen Einsatzes enthalten sein.
[IT-Landschaften 25]

\subsection{Abstraktionsniveau}
\label{ssc:Abstraktionsniveau}
Die Abstraktionsfähigkeit einer Modellierungssprache beschreibt ein Kriterium, worin die Fachterminologie der Domäne sich mit den Konzepten der Modellierungssprache möglichst decken sollte. In dem Bereich von Kommunikationsabläufen der Telekommunikation sind diese Begriffe eher technischer Natur und kommen aus einem Informatik-spezifischem Umfeld, welcher Fachsprache auf dem Niveau von Spezialisten voraussetzt. 

\subsection{Detailgrad}
\label{ssc:Detailgrad}
Der Detailgrad beschreibt wie Sachlich angemessen detailliert die Konstrukte eines Anwendungszwecks der Domäne sich darstellen lassen können. Dies beinhaltet die zu Modellierenden Modelle, Daten und zusätzlichen Informationen. Damit ist nicht gemeint, dass ein System in einer hohen Abstraktionsform eine eins zu eins Nachbildung 
aller technischen Prozesse des Informationssystems ausdrücken soll, sondern nur für die Zielgruppe angemessene und verständliche Konzepte. Eine formale Beschreibung der Prozesse ist demnach nicht gewünscht, da der Hauptaugenmerk auf dem Anwender liegt.

\subsection{Funktionalität}
\label{ssc:Funktionalität}
Anforderungen
Anpassung
Flexibilität

