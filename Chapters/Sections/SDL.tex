\section{\acf{SDL}}
\label{sc:SDL}
Die \ac{SDL} (SDL, engl. Spezifikations und Beschreibungssprache) ist eine von der \ac{ITU-T} (ITU-T, engl. Internationalen Telekommunikations Vereinigung für Standardisierung der Telekommunikation) standardisierte Modellierungssprache und wurde erstmalig 1976 definiert. Die aktuellste Version von \ac{SDL} ist SDL-2010, welche eine Überarbeitung der SDL-Version SDL-2000 aus dem Jahre 1999 ist. Wenn im folgenden von \ac{SDL} gesprochen wird, wird immer die Version \acs{SDL}-2010 gemeint. In den letzten Überarbeitungen wurde die Definition von \ac{SDL} um objekt-orientierte Aspekte erweitert und mit Sprachen wie UML und ASN.1 harmonisiert. \ac[SDL} wird zur Beschreibung von Telekommunikationssystemen und deren Abläufen, sowie für Protokoll Definitionen und verteilten Systemen eingesetzt. Der Hauptfokus von \ac{SDL} ist, sich einen genauen Überblick über das Verhalten genannter Systeme zu machen, wobei Eigenschaften mit anderen Techniken beschrieben werden müssen. Dazu konzentriert sich die Anwendung der Spezifikation unter anderem auf Echtzeitanwendungen, welche in der Spezifikation genauer zusammengefasst und beschrieben werden.

\subsection{Spracheigenschaften}
Die Eigenschaften der Sprachdefinition von \ac{SDL} sind folgend beschrieben[REC111_P.3]:
\begin{itemize}{
\item[Abstrakte Grammatik] Die abstrakte Grammatik von \ac{SDL} wird von einer abstrakten Syntax und  statischen Bedingungen beschrieben. Die Abstrakte Syntax kann entweder mit einer textbasierten Grammatik oder einem grafischen Metamodell erstellt werden.

\item[Konkrete Grammatik] Die konkrete Syntax wird durch eine grafische Syntax, statischen Bedingungen und Regeln für die grafische Syntax beschrieben. Beschrieben wird sie durch die erweiterte Backus-Naur Form. Wenn jedoch in der abstrakten Grammatik ein grafisches Metamodell verwendet wurde, ist es erlaubt dieses um kontrkete Eigenschaften zu erweitert und zu verwenden.

\item[Semantik] Die Semantik beschreibt ein Konstrukt,samt dessen Eigenschaften, Interpretation und dynamischen Bedingungen.

\item[Model] Ein Model gibt Notationen eine Abbildungsform, wenn diese keine direkte abstrakten Syntax besitzen.

\subsubsection{Metamodell}
Es werden Anstrengungen unternommen, jedoch existiert derzeit kein öffentlich zugängliches Metamodell von \ac{SDL}, welches alle Aspekte der Sprache in sich vereinigt. So hat die \ac{ITU-T} anstrengungen unternommen selbst ein Meta-Metamodell auf Grundlage von \ac{UML} zu erstellen, jedoch deckt diese Definition, welche auch SDL-UML genannt wird, nur Teile der Sprachdefinition von \ac{SDL} ab.


\subsection{Notation}
\subsection{Diagramarten}
