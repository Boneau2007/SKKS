\chapter{Fazit}


Eine strukturierte Liste von Beurteilungskriterien lässt sich daraus schwer ableiten.

\section{Referenz}
[Süttenbach und Ebert 1997]: Süttenbach, Roger; Ebert, Jürgen (1997): A Booch Metamodel. Institut für Informatik, Fachbereich Informatik, Universität Koblenz-Landau. Koblenz (Fachberichte Informatik, 5). Online verfügbar unter http://www.uni-koblenz.de/~ist/documents/Suettenbach1997ABM.pdf, zuletzt geprüft am 12.10.2015.
[Obermeier et al. 2014]: Obermeier, Stefan; Fischer, Herbert; Fleischmann, Albert; Dirndorfer, Max (2014): Geschäftsprozesse realisieren. Ein praxisorientierter Leitfaden von der Strategie bis zur Implementierung. 2., aktual. Aufl. Wiesbaden: Springer Vieweg (SpringerLink).
[Fischer et al. 2006]: Fischer, Herbert; Fleischmann, Albert; Obermeier, Stefan (2006): Geschäftsprozesse realisieren. Ein praxisorientierter Leitfaden von der Strategie bis zur Implementierung. 1. Aufl. Wiesbaden: Friedr.Vieweg und Sohn Verlag/GWV Fachverlage GmbH Wiesbaden (Aus dem Bereich IT erfolgreich nutzen).

[GoSt90] Goldstein, R.C.; Storey, V.C.: Some Findings on the Intuitiveness of Entity Relationship Constructs.
In: Lochovsky, F.H. (Ed.): Entity Relationship Approach to Database Design and
Query. Amsterdam: Elsevier 1990 \\
"[Hit95]" Hitchman, S.: Practitioner Perceptions on the Use of some Semantic Concepts in the Entity
Relationship Model In: European Journal of Information Systems, Vol. 4, 1995, pp. 31-40\\
"[Che81]" Checkland, P.: Systems thinking, systems practice. Chichester et al.: Wiley 1981\\